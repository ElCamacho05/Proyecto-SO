
\documentclass[11pt]{article}

% Idioma español
\usepackage[spanish]{babel}
\usepackage[utf8]{inputenc}

% Fuente tipo Arial
\usepackage{helvet}
\renewcommand{\familydefault}{\sfdefault}

% Interlineado 1.5
\usepackage{setspace}
\linespread{1.5}

% Márgenes personalizados
\usepackage[a4paper, top=2.5cm, bottom=2.5cm, left=3cm, right=3cm]{geometry}

% Color del texto
\usepackage{xcolor}
\color{black}

% Títulos personalizados: 13pt y en negrita
\usepackage{titlesec}
\titleformat{\section}
  {\bfseries\fontsize{13}{15}\selectfont} % formato: negrita y tamaño
  {\thesection}{1em}{}                   % numeración y espacio

\titleformat{\subsection}
  {\bfseries\fontsize{12}{14}\selectfont}
  {\thesubsection}{1em}{}

\titleformat{\subsubsection}
  {\bfseries\fontsize{11}{13}\selectfont}
  {\thesubsubsection}{1em}{}

\usepackage[T1]{fontenc}
\usepackage{amsmath}
\usepackage{graphicx}
\usepackage{geometry}
\usepackage{fancyhdr}
\usepackage{titlesec}
\usepackage{parskip}
\usepackage{listings}
\usepackage{xcolor}

\geometry{a4paper, margin=2.54cm}
\setlength{\parindent}{0pt}
\pagestyle{fancy}
\fancyhf{}
\rhead{Proyecto SO}
\lhead{Documentación Técnica}
\rfoot{Página \thepage}

\titleformat{\section}{\fontsize{13}{15}\bfseries}{\thesection}{1em}{}
\titleformat{\subsection}{\fontsize{13}{15}\bfseries}{\thesubsection}{1em}{}

\lstset{
    basicstyle=\ttfamily\small,
    keywordstyle=\color{blue},
    commentstyle=\color{gray},
    stringstyle=\color{orange},
    breaklines=true,
    showstringspaces=false,
    frame=single,
    numbers=left,
    numberstyle=\tiny\color{gray},
    captionpos=b
}

\begin{document}

\begin{titlepage}
    \centering
    \vspace*{1cm}
    {\Huge\bfseries Proyecto Final - Mini Sistema Operativo Académico \\[0.5cm]}
    {\Large Documentación del Proyecto\\[1.5cm]}
    \textbf{Integrantes:}
    \begin{flushleft}
        Camacho Castelán José Manuel \vspace{0.3cm}\\
        López Castillo Haziel \vspace{0.3cm}\\
        Martha Denisse Lara Xocuis \vspace{0.3cm}\\
        Pichal Pío Jose Armando \vspace{0.3cm}\\
        Xalanda Coyolt Karina Arlet \vspace{0.3cm}
    \end{flushleft}
    \vfill
    { \large H. Veracruz Ver. a \today}
\end{titlepage}

\tableofcontents
\newpage

\section{GestorPID.py}

\subsection{Descripción}
Clase utilizada para gestionar de manera incremental los identificadores únicos de procesos (PID). Es estática y permite asegurar unicidad.

\subsection{Código}
\begin{lstlisting}[language=Python, caption={GestorPID.py}]
class GestorPID:
    siguiente_pid = 1

    @classmethod
    def obtener_pid(cls):
        pid = cls.siguiente_pid
        cls.siguiente_pid += 1
        return pid
\end{lstlisting}

\subsection{Uso}
Llamar a \texttt{GestorPID.obtener\_pid()} devuelve el siguiente PID disponible.

\newpage

\section{Proceso.py}

\subsection{Descripción}
Representa un proceso del sistema operativo simulado. Hereda de \texttt{Thread} y ejecuta una función en paralelo.

\subsection{Código}
\begin{lstlisting}[language=Python, caption={Proceso.py}]
from threading import Thread
import time

class Proceso(Thread):
    def __init__(self, inicio, tamanio, func, pid):
        super().__init__()
        self.inicio = inicio
        self.tamanio = tamanio
        self.pid = pid
        self.func = func

    def run(self):
        if self.func:
            self.func(self.inicio, self.tamanio)
        else:
            print(f"[Proceso {self.pid}] No se asignó ninguna función.")

def tarea_ejemplo(inicio, tamanio):
    for i in range(5):
        print(f"[Proceso simulado] PID ocupando {tamanio} bloques desde {inicio}... ({i + 1}/5)")
        time.sleep(1)

def holamundo(inicio=None, tamanio=None):
    print("holamundo")
    time.sleep(1)
\end{lstlisting}

\subsection{Uso}
El constructor recibe dirección inicial de memoria, tamaño, función a ejecutar y PID. Se ejecuta con \texttt{start()}.

\newpage

\section{Memoria.py}

\subsection{Descripción}
Simula un espacio de memoria RAM donde se asignan bloques a procesos. Implementa el patrón Singleton.

\subsection{Código}
\begin{lstlisting}[language=Python, caption={Memoria.py}]
from kernel.Proceso import Proceso

class Memoria:
    _instance = None

    def __new__(cls, *args, **kwargs):
        if cls._instance is None:
            cls._instance = super().__new__(cls)
        return cls._instance

    def __init__(self, tamanio=100):
        if hasattr(self, '_initialized') and self._initialized:
            return
        self.tamanio = tamanio
        self.memoria = [0] * self.tamanio
        self.procesos = []
        self._initialized = True

    def asignar_memoria(self, pid, tamanio, func):
        libres = 0
        inicio = -1
        for i in range(len(self.memoria)):
            if self.memoria[i] == 0:
                if libres == 0:
                    inicio = i
                libres += 1
                if libres == tamanio:
                    for j in range(inicio, inicio + tamanio):
                        self.memoria[j] = pid
                    proceso = Proceso(inicio, tamanio, func, pid)
                    self.procesos.append(proceso)
                    print(f"Memoria asignada al proceso {pid} desde {inicio} hasta {inicio + tamanio - 1}")
                    return proceso
            else:
                libres = 0
        print("No hay suficiente memoria contigua disponible.")
        return None

    def liberar_memoria(self, pid):
        liberados = 0
        for i in range(len(self.memoria)):
            if self.memoria[i] == pid:
                self.memoria[i] = 0
                liberados += 1
        self.procesos = [p for p in self.procesos if p.pid != pid]
        print(f"Se liberaron {liberados} bloques del proceso {pid}.")

    def __str__(self):
        s = "Estado de la memoria:\n"
        for i in range(0, self.tamanio, 20):
            fila = self.memoria[i:i + 20]
            s += " ".join([str(b) if b != 0 else "-" for b in fila]) + "\n"
        s += "Procesos:\n"
        for p in self.procesos:
            s += f"Pid: {p.pid}, Inicio: {p.inicio}\n"
        return s
\end{lstlisting}

\subsection{Uso}
\begin{itemize}
    \item \texttt{asignar\_memoria(pid, tamanio, func)}: asigna bloques contiguos y crea el proceso.
    \item \texttt{liberar\_memoria(pid)}: libera todos los bloques ocupados por el proceso con ese PID.
    \item \texttt{str(mem)}: devuelve una representación de texto de la memoria.
\end{itemize}

\end{document}
